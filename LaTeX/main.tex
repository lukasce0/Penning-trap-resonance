
\documentclass[english,notitlepage,reprint,nofootinbib]{revtex4-1}

% Allows special characters (including æøå)
\usepackage[utf8]{inputenc}
% \usepackage[english]{babel}

% Note that you may need to download some of these packages manually, it depends on your setup.
% It may be usefult to download TeXMaker, because it includes a large library of the most common packages.

\usepackage{physics,amssymb}  % mathematical symbols (physics imports amsmath)
\usepackage{graphicx}         % include graphics such as plots
\usepackage{xcolor}           % set colors
\usepackage{hyperref}         % automagic cross-referencing
\usepackage{listings}         % display code
\usepackage{subfigure}
\usepackage{hyperref}
% imports a lot of cool and useful figure commands
% \usepackage{float}
%\usepackage[section]{placeins}
\usepackage{algorithm}
\usepackage{placeins}
\usepackage[noend]{algpseudocode}
\usepackage{subfigure}
\usepackage{tikz}
\usetikzlibrary{quantikz2}
% defines the color of hyperref objects
% Blending two colors:  blue!80!black  =  80% blue and 20% black
\hypersetup{ % this is just my personal choice, feel free to change things
    colorlinks,
    linkcolor={red!50!black},
    citecolor={blue!50!black},
    urlcolor={blue!80!black}}





\begin{document}
\include{amsmath}

\title{Simulation and resonance phenomena in Penning trap}  % self-explanatory
\author{Lukas Cernusak} % self-explanatory
\date{October 2023}                             % self-explanatory
\noaffiliation                            % ignore this, but keep it.

%This is how we create an abstract section.
\begin{abstract}
We created a simulation of a Penning trap with atomic ions, using electrostatic approximation. We compared two initial value methods, Forward Euler and Runge Kutta 4. We found an analytical solution for a single-particle system and visualized numerical solutions for a two-particle system. We compared Runge Kutta 4 and Forward Euler in both their relative error, which we found to be almost four orders of magnitude smaller for RK4, and error convergence rate which turned out to be 1.85 for RK4, and 1.40 for FE. In the end, we simulated an oscillating electric field, in order to find resonance frequencies at which particles are more likely to escape the trap. We found particle-to-particle interactions have surprisingly minor effects on the resonance frequencies.
    
\end{abstract}
\maketitle

%\textit{midpoint rule for integration}.
%\includegraphics[width=0.5\textwidth]{name.pdf}

% ===========================================
\section{Introduction}


A Penning trap is a device designed for containing charged particles. It requires both electric and magnetic fields, as achieving a bound state using only one of them is impossible. These particle traps are being used, among other things to store q-bits and antimatter. Interestingly they have played a crucial part in a recent experiment at CERN titled ``Observation of the effect of gravity on the motion of antimatter``[1], attempting to measure the gravitational interactions of antimater.

In order to simulate the Penning trap, we will work with both analytical and numerical solutions, requiring us to solve some initial value problems. 
Initial value algorithms undoubtedly form a fundamental basis of numerical analysis. Some of them are so simple that they are usually the first numerical algorithms students learn. In our simulation, we will be using mainly Rune Kutta 4, but initially, we will be comparing it to Forward Euler.

 
A bit later we will concentrate on something a bit different. The z-component of the electric field we will use is
$$  E_z =  - \frac{V_0}{2 d^2} 4 y . $$
This is analogous to Hook's law, meaning the electric field causes harmonic oscillations along the z-axis. If we then change the potential difference $V_0$ to a harmonically oscillating one, we expect to observe some resonance frequencies, that will cause more particles to escape the trap. However, in the actual simulation, we will also have to consider particle-to-particle interactions and the forces on the xy-plane. These are somewhat more complicated as they depend on both the electric and magnetic fields. We will therefore simulate a system, vary the electric oscillation both in amplitude and frequency, and study the number of particles that will escape our Penning trap.






\section{Methods}\label{sec:methods}
\subsection{Background}

First of all, we need to know the relevant forces acting on the particles. We will assume that inside the trap there is a homogeneous magnetic field parallel to the z-axis
\begin{equation}
 \vec{B} = B_0 \hat{k}.
\end{equation}
Next, we have an electric field. A Penning we will consider has an equally charged top and bottom plate and one central ring with opposite polarity. Which side is positive and which is negative depends on the charge of the contained particle. We will work with positive ions, meaning the plates will be positively charged and the ring will be negative. Here the actual field depends on the structure of the Penning trap, but we will be using
\begin{equation}
 \vec{E}(x,y,z) = \frac{V_0}{2 d^2} \Big( 2x \hat{i} + 2y \hat{j} - 4y \hat{k} \Big).
\end{equation}
Here $d$ is the characteristic dimension, which is correlated to the distances between the electrodes, and $V_0$ is the potential difference over them. Initially, we will keep the potential constant, but later we will introduce a harmonically oscillating term.
\begin{equation}
V_0 \quad \rightarrow \quad V_0 (1 + f \, cos(\omega_V \, t))
\end{equation}
We will assume both the electric and magnetic field is zero outside the trap. Although at no point will the physical properties of particles that have left the traped be important to us.

The last type of interaction we will include, is the one between the particles themselves. Here it's crucial to mention that the speeds we will work with are on the order of tens of meters per second, which is much less than the speed of light. We will therefore use the electrostatic approximation, with other words ignoring the magnetic interactions between particles. By Coulomb's law, the electric caused by N charged particles is
\begin{equation}
 \vec{E} = k_e \sum_{j=1}^N q_j \frac{\vec{r} - \vec{r}_j}{|\vec{r} - \vec{r}_j|^3} .
 \end{equation}
Where $k_e$ is the Coulombs constant, $q$ is the charge and $\vec{r}$ is the position vector. Finally, the total force on a particle is
\begin{equation}
 \vec{F} = q \vec{E} + q \vec{v} \times \vec{B} .
\end{equation}

\subsection*{Analytical}
Now we wish to find an analytical solution to the motion of the particles. A general solution to the N particle system do not exist. Therefore we will in this sub-section, work with only one particle. We start by finding the equations of motion. Using eq. (1), (2), (5), and Newton's second law, we get the following.
\begin{equation}
\ddot{x} - \omega _0 \dot{y} - \frac{1}{2} \omega _z ^2 x = 0
\end{equation}
\begin{equation}
\ddot{y} + \omega _0 \dot{x} - \frac{1}{2} \omega _z ^2 y = 0
\end{equation}
\begin{equation}
\ddot{z} + \omega _z ^2 z = 0
\end{equation}
Here we define two new variables $ \omega_0 \equiv \frac{q B_0 ^2}{m}$ and $ \omega _z ^2 \equiv \frac{2qv_0}{md^2} $, where m is mass. Eq. (8) is an ordinary second-order linear differential equation, with general solution
\begin{equation}
    z(t) = c_1 sin( \omega_z t) + c_2 cos( \omega_z t).
\end{equation}
Eq. (6) and (7) are partial differential equations, but under closer inspection we find they are coupled. If we define a new function $f(t) \equiv x(t) + iy(t)$, by using eq. (6) and (7) the $f(t)$ may be rewritten to
$$ \ddot{f} + i \omega_0 \dot{f} - \frac{1}{2} \omega_z ^2 f = 0 .$$
A general solution to this equation is
\begin{equation}
f(t) = A_+ e^{-i(\omega_+ t + \phi_+)} + A_- e^{-i(\omega_- t + \phi_-)}.
\end{equation}
Here $A_{\pm}$ and $\phi _{\pm}$ are positive constants dependent on initial values and $\omega _{\pm}$ is
$$ \omega_{\pm} = \frac{\omega_0 \pm \sqrt{ \omega_0 ^2 - 2 \omega _z^2}}{2}. $$
Armed with this solution we may set conditions for the bound particle states in the trap. What a bound state requires, is that neither x or y diverge to infinity, formally we may write $lim _{t \rightarrow \infty} \,|f(t)| < \infty $. We observe that the exponents in (10) needs to be imaginary. Note that if they were complex numbers, their real parts would have opposite signs, always making one of the components divergent (self contradiction). This boils down to the requirement for $\omega_{\pm}$ to be real, which leads to
\begin{equation}
    \frac{q}{m} \geq \frac{2V_0}{B_0^2d^2}
\end{equation}
An other thing we may do with the solution (10), is to explore possible displacements on the xy-plane relative to the origin. Thanks to the equivalent properties of the complex plane and Cartesian vectors, we may just say
$$ |f(t)| = \sqrt{A_+^2 + A_-^2 + 2 A_+ A_- cos(\beta)} .$$
Where the angle is defined as $\beta = t(\omega_- - \omega_+) + \phi_- - \phi_+$. This means that our minimal and maximal of-sets are:
\begin{equation}
|f|_{min} = |A_+ - A_-|
\end{equation}
\begin{equation}
|f|_{max} = A_+ + A_-
\end{equation}
In order to find errors in our numerical solutions, we want an analytical one we can compare to. Now it would be handy to come up with a specific solution of the initial value dependant factors ($A_{\pm}$ and $\phi_{\pm}$). If we chooses initial conditions as $\vec{r}_0 = (x_0, 0, z_0)$ and $\vec{v} = (0,v_0,0)$, the solution on z-axis becomes
\begin{equation}
z(t) = z_0 cos(\omega_z t).
\end{equation}
On the xy-plane, we get the following definitions for the factors.
\begin{equation}
A_{\pm} = \pm \frac{v_0 + \omega_{\mp} x_0}{\omega_- - \omega_+}
\end{equation}
\begin{equation}
\phi_{\pm} = 0
\end{equation}

\subsection*{Algorithms}

Before we start working on algorithms, we need to discretize our equations. We choose the initial time to be zero, and the rest is as follows.
$$ t \rightarrow t_i \in \{ 0, h, 2h \cdot \cdot \cdot , (n+1)h \} $$
\begin{equation}
    \vec{r} \rightarrow \vec{r}_i \in \{ \vec{r}_0, \vec{r}_1, \vec{r}_2, \cdot \cdot \cdot, \vec{r}_{n+1} \}
\end{equation}
$$ \vec{v} \rightarrow \vec{v}_i \in \{ \vec{v}_0, \vec{v}_1, \vec{v}_2, \cdot \cdot \cdot, \vec{v}_{n+1} \} $$
Variable n describes the number of steps. Reason for having $n+1$ values, is that the set includes the initial value too.

We will use two initial value methods in our simulations. First one is Forward Euler (FE). This is a rather easy method to deduce. We just need to Taylor expand the discretized $x$ around zero, and for $x_{i+1}$, giving us
\begin{equation}
    x_{i+1} = x_i + \dot{x}_i h  + \mathcal{O}(h^2).
\end{equation}
This means the FE has a local error on the order of $\mathcal{O}(h^2)$ and global error of order $\mathcal{O}(h)$. Although we did Taylor expand over time, nothing is preventing $x_i$ from also being a function position or velocity, as it will be with our simulation. 

The second algorithm we will implement, is the Runge Kutta 4 (RK4). This algorithm is a more sophisticated version of Predictor Corrector. What makes Predictor-Corrector methods special, is that they do work with gradients not only past, but also in the future (predicted). This is straightforward for only time-dependent slopes, however, if the slope is also room-dependent, we need to create a prediction of the position before making prediction of the gradient. To make this future prediction, we will use FE.

We will not proofing RK4 in this report. The method has a local error of $\mathcal{O}(h^5)$ and global of $\mathcal{O}(h^4)$. The order (in this case 4) of the algorithm refers to the number of slopes at different locations or times one step of RK4 considers. For the order of four, the slopes are as follows. 
$$ K_1 = h \dot{x}\Big(t_i,x_i \Big) $$
$$ K_2 = h \dot{x}\Big(t_i + \frac{h}{2},x_i + \frac{K_1}{2}\Big) $$
\begin{equation}
K_3 = h \dot{x}\Big(t_i + \frac{h}{2},x_i + \frac{K_2}{2}\Big)
\end{equation}
$$ K_4 = h \dot{x}\Big(t_i + h,x_i + K3 \Big) .$$
The final step that will be used in the simulation is
\begin{equation}
    x_{i+1} = x_i + \frac{1}{6} \Big[ K_1 + 2K_2 + 2K_3 + K_4 \Big].
\end{equation}
The reason for the double importance of $K_2$ and $K_3$ over $K_1$ and $K_4$ follows from the Simpsons integration rule. However, it's also intuitive, as $K_2$ and $K_3$ times are between $t_i$ and $t_{i+1}$, instead of being at them like the two remaining are.

We need some tools for comparing the numerical errors. We will be comparing the errors directly, as we are provided with analytical solution. Therefore, we will use relative errors:
\begin{equation}
    \nu = \frac{|\vec{r}_{i,exact} - \vec{r}_i|}{|\vec{r}_{i,exact}|}.
\end{equation}
We will too get use for the error convergence rate. The simulation will be run with four different step sizes, so the error convergence rate is
\begin{equation}
    r_{err} = \frac{1}{3} \sum_{k=2}^4 \frac{log(\Delta_{max,k}/\Delta_{max,k-1})}{log(h_k/h_{k-1})}.
\end{equation}
Where $h_k$ refers to the step size of the simulation $k$ and the $Delta_{max,k}$ is the maximum absolute error on the entire interval of a simulation, formally
$$ \Delta _{max,k} = max_i |\vec{r}_{i,exact} - \vec{r}_i| . $$

\subsection*{Simulation}
First of all, we need to mention the physical quantities of our Penning trap. Since we are working with individual atoms, it is useful to work with the units of $e$ for charge and $u$ for mass. Due to the magnitude of the distances, it also comes in handy to work with time in $\mu s$ and distances in $\mu m$. The magnetic field flux density is $1T$, the electric potential constant $V_0$ is $25mV$, and the characteristic distance $d$ is $500 \mu m$.
We will be simulating $Calcium^+$ ions. As 97\% of natural Calcium is the $^{40} _{20} Ca$ isotope, our particles will have the mass of $40u$ and charge $1e$. The initial values we will use in simulations with one particle are $\vec{r}_0 = (20,0,20)\mu m$ and $\vec{v}_0 = (0,25,0)m/s$. Note that for these values we may use the specified analytical solution (14), (15) and (16).
For simulation with two particles, the first particle will be as specified above and the second particle will have initial conditions $\vec{r}_0 = (25,25,0)\mu m$ and $\vec{v}_0 = (0,40,5)m/s$

Now a bit about the simulations we will be performing. (i) First we will observe 
the motion and velocities of the particles, and also the effects of particle-to-particle (p-t-p) interactions. To do so, we will run four simulations with two particles, both p-t-p interacting and non-interacting, using RK4 over a time period of $50 \mu s$ and with 1000 steps. 
(ii) Second, we will explore the errors. We will do nine simulations with one particle, using both FE and RK4 over $50 \mu s$ and steps ranging from 1000 to 32000. 
(iii) Lastly, we will be exploring the resonance phenomena. We will do multiple simulations using again RK4, but this time with 100 particles, oscillating electric field, over a time period of $500 \mu s$ and with 40000 steps. The particles will be assigned pseudorandom initial values, from a normal distribution. Here we will keep track over the electric oscillation frequencies, its magnitudes and a fraction of particles still trapped at the end of the simulations.


\subsection*{Tools}
All the plots have been created using Pythons \texttt{matplotlib} library in cooperation \texttt{numpy}. The particle vectors were represented using \texttt{armadillo} vector. Further, the pseudorandom initial values of the 100 particles have been chosen using \texttt{armadillo} randn function and using value 123 as seed.



\section{Results and discussion}\label{sec:results_and_discussion}

\subsection*{Visualized Motion}


We solve the eq. (6), (7), and (8) (no p-t-p) numerically, using RK4 described in eq. (19) and FE described in eq. (18). Plotting the positions on the xy-plane, we get figure~\ref{fig:xy_no}. In the plot we observe a periodic motion. As the solution can be obtained analytically, a brief check confirms our results are trustworthy. Now that we have illustrated periodic motion on the not so obvious xy-plane, the resonance phenomena is not limited along the z-axis.
\FloatBarrier
\begin{figure}[!htb]
    \centering %Centers the figure
    \includegraphics[scale=0.4]{xy_no.pdf} %Imports the figure.
    \caption{Movement of two non-interacting particles on the xy-plane inside a Penning trap. Solved using RK4 with 1000 steps over $50 \mu s$}
    \label{fig:xy_no}
\end{figure}
\FloatBarrier

If we now add the Coulomb interactions described in eq. (4), to the electric field in equation (5), we get an additional term in our differential equations of motion ((6) and (7)). Solving using RK4, and plotting the solution we get figure~\ref{fig:xy_yes}. Here we see the particle-to-particle interactions are by no means negligible, as there is an obvious difference between the plots.

\FloatBarrier
\begin{figure}[!htb]
    \centering %Centers the figure
    \includegraphics[scale=0.4]{xy_yes.pdf} %Imports the figure.
    \caption{Movement of two interacting particles on the xy-plane inside a Penning trap. Solved using RK4 with 1000 steps over $50 \mu s$}
    \label{fig:xy_yes}
\end{figure}
\FloatBarrier

Now we wish to see the whole motion, so we plot a three-dimensional. In hopes of getting a better grasp of the movement. Figure~\ref{fig:3d_no} is the result of a the simulations without p-t-p interactions and figure~\ref{fig:3d_yes} is the solutionwith this force. We now observe a disadvantage of three dimensional plots. They lack a sense of depth, making our solution almost seem chaotic, although as we a have seem it isn't.


\FloatBarrier
\begin{figure}[ht]
    \centering %Centers the figure
    \includegraphics[scale=0.4]{3d_no_interactions_84.pdf} %Imports the figure.
    \caption{Movement of two non-interacting particles in three dimensions inside a Penning trap. Solved using RK4 with 1000 steps over $50 \mu s$}
    \label{fig:3d_no}
\end{figure}
\FloatBarrier
\FloatBarrier
\begin{figure}[ht]
    \centering %Centers the figure
    \includegraphics[scale=0.4]{3d_interactions_84.pdf} %Imports the figure.
    \caption{Movement of two interacting particles three dimensions inside a Penning trap. Solved using RK4 with 1000 steps over $50 \mu s$}
    \label{fig:3d_yes}
\end{figure}
\FloatBarrier

Now wish to see the phase space of this system. We will plot one position component along the one axis of the plot and the velocity of the same component along the second. Since  x and y are symmetric in our simulations, there is no reason to be studying both of them simultaneously. We choose therefore the x and z axis. Figure~\ref{fig:phase_x_no} shows solution without p-t-p interactions along the x-axis, and Figure~\ref{fig:phase_z_no} shows the same simulation along the z-axis. Both of these graphs are linear with slope 1, meaning the values that are ploted against each other are equal. It is not surprising that the graph is stable, but it's surprising that the components should be equal. It is possible that I have done a mistake. To know the answer we would need to dive much deeper into this specific values.


\FloatBarrier
\begin{figure}[htbp]
   \centering %Centers the figure
    \includegraphics[scale=0.4]{phase_x_no.pdf} %Imports the figure.
    \caption{The velocity-position plot of two non-interacting particles along the x-axis. Solved using RK4 with 1000 steps over $50 \mu s$}
    \label{fig:phase_x_no}
\end{figure}
\FloatBarrier

\FloatBarrier
\begin{figure}[ht]
    \centering %Centers the figure
    \includegraphics[scale=0.4]{phase_z_no.pdf} %Imports the figure.
    \caption{The velocity-position plot of two non-interacting particles along the z-axis. Solved using RK4 with 1000 steps over $50 \mu s$}
    \label{fig:phase_z_no}
\end{figure}
\FloatBarrier

Same graph as the two previous, only with p-t-p interactions can be seen in figure~\ref{fig:phase_x_yes} for the x-axis, and in figure~\ref{fig:phase_z_yes} for the z-axis. On these two plots we see a much more complicated behaviour. The p-t-p interactions of course cause different velocities at different positions, but there is likely also a considerable energy and momentum exchange between the particles. As a result, although we are able to see some patterns in motions on the xy-plane, there are almost none in this phase diagram. It's interesting that the x-axis phase space seems to have more fluctuations, but has somewhat more repetitive behavior, in comparison to the z-axis.

\FloatBarrier
\begin{figure}[htbp]
    %\centering %Centers the figure
    \includegraphics[scale=0.4]{phase_x_yes.pdf} %Imports the figure.
    \caption{The velocity-position plot of two interacting particles along the x-axis. Solved using RK4 with 1000 steps over $50 \mu s$}
    \label{fig:phase_x_yes}
\end{figure}
\FloatBarrier

\FloatBarrier
\begin{figure}[ht]
    \centering %Centers the figure
    \includegraphics[scale=0.4]{phase_z_yes.pdf} %Imports the figure.
    \caption{The velocity-position plot of two interacting particles along the z-axis. Solved using RK4 with 1000 steps over $50 \mu s$}
    \label{fig:phase_z_yes}
\end{figure}
\FloatBarrier





\subsection*{Errors}

Here we are unfortunately forced to work with only one particle, as we wish to compare the numerical and analytical solutions. First of all, let's make a demonstrative visualization of the instability. Using the specific solution for $z(t)$ in eq. (14), and plotting it as a function of time and the same plot as numerical solutions obtained by both the RK4 and FE, we get figure~\ref{fig:zt}. Note that the steps in FE simulation have been chosen specifically such that it's possible to observe the instability without the particle escaping the trap within our time interval. In comparison to FE, we can't observe any error by eye for the RK4 solution.

\FloatBarrier
\begin{figure}[ht]
    \centering %Centers the figure
    \includegraphics[scale=0.4]{theor_vs_analytic_81.pdf} %Imports the figure.
    \caption{Position of one particle along the z-axis as a function of time over $50 \mu s$. The plot includes an analytical solution, a RK4 solution with 1000 steps, and a FE solution with 2000 steps.}
    \label{fig:zt}
\end{figure}
\FloatBarrier

In order to be able to compare these methods, we will plot their relative error eq. (21) as a function of time. In figure~\ref{fig:error_RK4} we see result for RK4 and in figure~\ref{fig:error_FE} for FE. This graph looks trustworthy, as we would expect the errors to be zero at the initial values, and grow for over time. The differences in relative error, theses two methods make are on the order of almost four magnitudes. This is a incredible difference. With that being said, here we see that the error in RK4 is also growing over time. 

To compare the error convergent rate for these two methods, we will use eq. (22). Using values we have obtained, we get $r_{err,RK4} = 1.85$ and $r_{err,FE} = 1.40$. This means that the error in RK4 should grow slower than the one for FE. This is tricky to observe directly in the plot, as the axis ratios are different. It seems like RK4 is superior, but we need to remember that RK4 is a lot more computationally demanding. To figure out by how much, we could e.g. count FLOPs, or measure time duration withing the code for the individual methods. This we unfortunately will not explore this time, but would be for sure actual and interesting. 

\FloatBarrier
$$ \\ $$
$$ \\ $$

\begin{figure}[ht]
    \centering %Centers the figure
    \includegraphics[scale=0.4]{resonance_all_RK4.pdf} %Imports the figure.
    \caption{Relative error as a function of time for one particle. Solved using RK4.}
    \label{fig:error_RK4}
\end{figure}
\FloatBarrier
\FloatBarrier
\begin{figure}[ht]
    \centering %Centers the figure
    \includegraphics[scale=0.4]{resonance_all_FE.pdf} %Imports the figure.
    \caption{Relative error as a function of time for one particle. Solved using FE.}
    \label{fig:error_FE}
\end{figure}
\FloatBarrier




\subsection*{Resonance}

Now we will explore the resonance frequencies of the particles in an oscillating electric field as described in (3). In figure~\ref{fig:resonance_all} we are plotting the fraction of particles left of originally 100 after $500 \mu s$, as a function of angular frequency $\omega_v$ and for different amplitudes $f$ of the electric field. The simulation does not include particle-to-particle interactions. We see that at some frequencies, almost all particles do leave the trap, but at other, almost no particles does.

The existence of these frequencies is not really surprising, as we have seen the periodical movements of non interaction particles.

The resolution of the x-axis could be better, as we have some resonance frequencies that only last for one or two points on the x-axis. We might be able to see more of narrow frequencies if we had a better resolution., as some may be narrower than the steps on the x-axis. 

\FloatBarrier
\begin{figure}[ht]
    \centering %Centers the figure
    \includegraphics[scale=0.4]{resonance_all.pdf} %Imports the figure.
    \caption{The fraction of trapped particles left after $500 \mu s$ with an oscillation amplitude of f, as a function of oscillation frequency. Initially, there were 100 particles and there were no particle interactions.}
    \label{fig:resonance_all}
\end{figure}
\FloatBarrier


In the end, we want to see whether particle-to-particle interactions have a considerable effect on these frequencies. Simulating particle interactions for many simulations is much more computationally demanding than the previous simulations. We will therefore zoom in on one of the frequencies. 

Figure~\ref{fig:zoom_no} plots the result for non-interacting particles and figure~\ref{fig:zoom_yes} shows interacting particles. As we can see from the plots, it is slightly different. It seems like the whole graph is moved slightly to the right side, yet another reason to include a greater resolution of the x-axis. However, the difference is almost negligible, if we compare it to the frequency interval where all the particles have left. 

It is really surprising that the difference is so small, as we have seen the particle-to-particle interactions have clear effects for two particles. When we had 100 particles, the electric repulsion between them was much higher (at least for those close to the edges) than for the two-particle system. It would be interesting to see what would happen for even more particles, where the particle-to-particle forces would dominate.



\FloatBarrier
\begin{figure}[ht]
    \centering %Centers the figure
    \includegraphics[scale=0.4]{zoom_no.pdf} %Imports the figure.
    \caption{The fraction of trapped particles left after $500 \mu s$ with an oscillation amplitude of f, as a function of oscillation frequency. Initially, there were 100 particles with no particle interactions. Effectively zoomed-in slightly more detailed version of plot \ref{fig:resonance_all}.}
    \label{fig:zoom_no}
\end{figure}
\FloatBarrier
\FloatBarrier
\begin{figure}[ht]
    \centering %Centers the figure
    \includegraphics[scale=0.4]{zoom_yes.pdf} %Imports the figure.
    \caption{The fraction of trapped particles left after $500 \mu s$ with an oscillation amplitude of f, as a function of oscillation frequency. Initially, there were 100 particles and there were particle interactions.}
    \label{fig:zoom_yes}
\end{figure}
\FloatBarrier






\section{Conclusion}\label{sec:conclusion}

We compared two RK4 and FE in a Penning trap simulation for positive Calcium ions. The RK4 turned out to have almost four orders of magnitude. The error convergence rate was 1.85 for RK4 and 1.40 for FE. All of our simulations, except for the ones specified on numerical errors, we performed using RK4. We found electric particle-to-particle interactions to have a clear effect even in a two-particle system. We have investigated resonance frequencies using time dependant electric field and a trap with 100 particles. Rather interestingly the particle-to-particle interactions didn't seem to have a strong effect on the frequencies. 


\section{References}
[1] Observation of the effect of gravity on the motion of antimatter, 27 September 2023, Nature,
\url{https://www.nature.com/articles/s41586-023-06527-1}



The code used is available in the GitHub repository at the following URL %instructions 
\url{https://github.uio.no/lukasce/FYS3150_Project3.git}

%
%

\onecolumngrid
\bibliographystyle{apalike}
%\bibliographystyle{unsrt}
\bibliography{ref} %For some reason this doesn't work for this raport (works for other I have tried. I suspect it may have to do with the FloatBarriers which I need or else my plots gets thrown at the end of the file.





\end{document}

